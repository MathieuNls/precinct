
\documentclass[conference]{IEEEtran}
\usepackage{cite}
\usepackage{graphicx}
\usepackage{amsmath}
\usepackage{enumerate}
\usepackage{xcolor}
\usepackage{pgfplots}
\usepackage{tikz}

\definecolor{bblue}{HTML}{4F81BD}
\definecolor{rred}{HTML}{C0504D}
\definecolor{ggreen}{HTML}{9BBB59}
\definecolor{ppurple}{HTML}{9F4C7C}

% correct bad hyphenation here
\hyphenation{op-tical net-works semi-conduc-tor}


\begin{document}

\title{PRECINCT: An Incremental Algorithm to Prevent Clone Insertion}


\author{\IEEEauthorblockN{Mathieu Nayrolles, }
\IEEEauthorblockA{Software Behaviour Analysis (SBA) Research Lab\\
ECE, Concordia University\\
Montreal, Canada\\
m\_nayrol@ece.concordia.ca}
\and
\IEEEauthorblockN{Abdelwahab Hamou-Lhadj}
\IEEEauthorblockA{Software Behaviour Analysis (SBA) Research Lab\\
ECE, Concordia University\\
Montreal, Canada\\
abdelw@ece.concordia.ca}}

% make the title area
\maketitle

% As a general rule, do not put math, special symbols or citations
% in the abstract
\begin{abstract}
  Software clones are considered harmful in software maintenance and evolution. However, after a decade and a half of research, only few approaches have targeted the prevention aspect of clones detection.
  In this paper, we propose a novel approach named PRECINCT (PREventing Clones INsertion at Commit Time). PRECINCT focuses on near-miss software clones are copied -- or reinvented -- fragments where minor to extensive modifications have been made and more specifically on detecting at commit time by means of pre-commit hooks.
  Efficiently detecting near-miss software clones at commit time might call for further refactoring or simply hint developers that they reinvented one piece of code.
  We apply and validate PRECINCT in terms of precision and recall on seven systems developed independently with a wide range of technologies, size and purposes.
  The validation demonstrates that our approach detects near-miss software clones before they reach the source version system with a 100\% precision and a 93\% recall.



\end{abstract}


\IEEEpeerreviewmaketitle




\bibliographystyle{IEEEtran}
\bibliography{library.bib}


\end{document}
